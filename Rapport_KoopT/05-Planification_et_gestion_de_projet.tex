\chapter{Planification et gestion de projet}
\label{sec:Planification et gestion de projet}

\section*{Introduction}

La planification et la gestion de projet au sein de GLOOKAL pour l’application KoopT a été basé sur deux  principaux outils JIRA et SVN.
tout au long du stage on suivaient l’organisation suivante :

- Faire un Update (SVN) au début de chaque journée 

- Faire un Commit (SVN) en fin de chaque journée ou dès qu’une tâche est aboutie avec un message qui contient le numéro de la tâche en cours dans JIRA

- Les tâches à réaliser sont sous forme des fiches JIRA qui porte le nom du rapporteur (Maître de stage) et le nom du responsable ( la personne chargé de réaliser la tâche)

- Ajouter un journal de travail dans JIRA on décrivant le déroulement et le travail réalisé  pour le développement de la tâche ainsi que le temps consacré à sa réalisation.

- Changer le statut de la fiche JIRA de "En cours" à "A valider" une fois la tâche est aboutie.

\addcontentsline{toc}{section}{Introduction.}


\section{JIRA ATLASSIAN}

Jira est un système de suivi de bugs, un système de gestion des incidents, et un système de gestion de projets développé par Atlassian.
Jira est un gestionnaire des demandes, une demande peut être un bug, une anomalie, un incident, une demande d’intervention…  JIRA est un outil de suivi d’activités.
JIRA permet la Planification des projets en créant des user stories et des tickets, en planifiant des sprints et en affectant les à l’ensemble de l’équipe de développement.
 
 
Les demandes ou les tâches KoopT sont gérées à travers quatre status principaux : 

- A Faire : contient les tâches KoopT planifiées à faire prochainement.

- En Cours :  contient les tâches KoopT en cours de développement.

- A Valider : contient les tâches KoopT qui sont abouties mais pas encore validées.

- Résolues : contient les tâches KoopT abouties et validées. 

\section{Subversion (SVN)}

Subversion (en abrégé svn) est un logiciel de gestion de versions, distribué sous licence Apache et BSD. Il a été conçu pour remplacer CVS. Ses auteurs s'appuient volontairement sur les mêmes concepts (notamment sur le principe du dépôt centralisé et unique) et considèrent que le modèle de CVS est bon, seule son implémentation est perfectible.

Subversion fonctionne donc sur le mode client-serveur, avec :

- un serveur informatique centralisé et unique où se situent :
les fichiers constituant la référence (le « dépôt » ou « référentiel », ou « repository » en anglais),
un logiciel serveur Subversion tournant en « tâche de fond » ;

- des postes clients sur lesquels se trouvent :
les fichiers recopiés depuis le serveur, éventuellement modifiés localement depuis,
un logiciel client, sous forme d'exécutable standalone (ex. : SmartSVN) ou de plug-in (ex. : TortoiseSVN, Eclipse Subversive) permettant la synchronisation, manuelle et/ou automatisée, entre chaque client et le serveur de référence
 
