\chapter*{Introduction}
\label{chap:introduction}
%\minitoc

84\% des français surfent quasiment quotidiennement sur leur smartphone, et ont au moins créé un rendez-vous sur un agenda électronique ; 71\% des français ont déjà utilisé leur mobile pour chercher des informations, avant d’acheter ou d’avoir recours à une prestation de service; 9 Français sur 10 ont déjà eu recours à une application dite de consommation collaborative.

En observant ces statistiques, on constate qu’un utilisateur a un besoin universel et trivial c’est de planifier, trouver, réserver rapidement et facilement des RDV avec quiconque, quel que soit le contexte, personnel ou professionnel.


L’application KOOPT s’inscrit dans le récent développement des conciergeries digitales et collaboratives, ainsi que des services de prise de rendez-vous. Pour toutes ces solutions, la promesse utilisateur est : efficacité, gain de temps, exhaustivité de l’offre, économies d’échelle et financières. Toutefois, ces outils, essentiellement basés sur des plateformes d’échange centralisées, rencontrent de nombreux freins à l’adoption massive : la sectorisation (verticalisation de l’offre), le manque d’automatisation (planification d’un même RDV pour plusieurs participants), l’adaptabilité et la flexibilité (gestion des aléas ou des contraintes en temps réel et de façon simultanée/parallèle), la communication ad hoc entre les acteurs... ou encore la confidentialité des échanges ou des données personnelles. 
				
Dans le cadre de ma première année de Master MIAGE (Méthodes informatiques appliquées à la gestion des entreprise) à l'Université de Paris ouest Nanterre la Défense, j'ai souhaité réaliser mon stage dans une entreprise répondant à ces enjeux par une application mobile Android tout en me formant sur le développement des applications Android (Programmation Mobile)  que ma formation propose comme débouché. 					
J'ai donc effectué un stage du 3 Avril au 31 juillet au sein de la société GLOOKAL dont l'activité est la conception et la création de l’application KOOPT. 
				
L'entreprise française GLOOKAL, créée très récemment en 2016 a pu étudier et mettre en place un tout nouveau concept (B-ADSc) pour la conception de l’application KOOPT qui se base sur l’intelligence artificielle. j'ai voulu intégrer l’équipe technique de GLOOKAL pour pouvoir découvrir les bases de ce concept ainsi de m’approfondir et m'enrichir dans la programmation des applications Android. 
 
C’est dans cette optique que s’inscrit mon projet de Master 1 et le présent rapport présente les fondements théoriques et les aspects techniques nécessaires à l’implémentation de la  nouvelle solution KOOPT.
		
Ce présent rapport est divisé en 7 chapitres :
 
Chapitre 1 : Cadre général de stage, présente l’organisme d’accueil ainsi que  le cadre général du stage.
\newline
Chapitre 2 : État de l’art, présente les différentes notions et technologies nécessaires pour le développement de KOOPT.
\newline
Chapitre 3 : Étude préalable de projet et implémentation, présente l’application KOOPT et ses objectifs.
\newline
Chapitre 4 : Planification et gestion de projet, présente les outils utilisés pour la planification et la gestion du projet.
\newline
Chapitre 5 : Compatibilité sous Android, présente la mission de gestion de compatibilité de KOOPT sur toutes les versions Android.
\newline
Chapitre 6 : Amélioration technique de KOOPT, présente les  différents améliorations de KOOPT durant le stage. 
\newline
Chapitre 7 : Débogage du Moteur KOOPT, présente la mission de débogage du KOOPT à travers les Logs.

\addcontentsline{toc}{chapter}{Introduction.}

%%% Local Variables: 
%%% mode: latex
%%% TeX-master: "isae-report-template"
%%% End: 
