\chapter{Cadre général de stage}
\label{chap:Cadre general de stage}

\section*{Introduction}

Le premier chapitre de ce rapport donne un aperçu général sur l’organisme d’accueil ainsi que sur l’équipe du projet au sein de l’entreprise et décrit le context général du stage et ses objectifs.

\addcontentsline{toc}{section}{Introduction.}

\section{Présentation de GLOOKAL}

GLOOKAL SAS, créée en octobre 2016 par 5 associés fondateurs, développe une solution d'intelligence Décisionnelle Artificielle, embarquée dans une App mobile à destination des segments B2B, B2C et C2C. KOOPT est le 1er smart-assistant ‘BYOD’, 100\% intelligent, automatique, collaboratif et transparent. Personnel, il est embarqué sur chaque smartphone. Fonctionnant en pair-à-pair, c’est la première App d’intermédiation extensible à l’infini, qui ne partage aucune information personnelle et respecte réellement la vie privée de ses utilisateurs. 
À l’heure actuelle, le capital actuel (81.400,00 €) est détenu à 93,32\% par les associés fondateurs, le reste étant détenu par des investisseurs de type LOVE MONEY. GLOOKAL déjà développé un premier MVP (Produit Minimum Viable), sous Android, qui permet de tester le marché et la solution.
 \begin{itemize}
     \item Raison sociale : GLOOKAL SAS 
     \item Date de création : 26/10/2016
     \item Effectif : 2
     \item SIREN : 823 374 780
     \item Secteur d’activité : Numérique
     \item Adresse : 88 Avenue des ternes 75017 PARIS 
 \end{itemize}


\section{Business Model}
 \begin{itemize}
     \item Modèle projet (court-terme) : Implémentation ‘sur mesure’ de KOOPT pour répondre aux usages privés des entreprises (marque blanche). Ces projets permettront de garantir une traction et de générer, sur le court terme, du chiffre d’affaires. Rémunération forfaitaire et adaptée selon le profil du projet.
     \item Modèle Freemium (moyen-terme) : KOOPT sera disponible gratuitement pour le marché du grand-public (C2C), et en parallèle une version premium sera proposée aux offreurs de services professionnels pour un montant annuel de 99€ HT avec des fonctionnalités avancées.		
     \item Modèle redistributif (long-terme) : lorsqu’un professionnel sera recommandé par un utilisateur de KOOPT (agissant comme intermédiaire), une commission de 1 € sera prélevée sur ses honoraires et répartie entre GLOOKAL et l’intermédiaire. 
 \end{itemize}
		



 \section{Equipe de projet}
 
  \begin{itemize}
      \item Philippe Gautier – Président (Associé) : Provisoirement en charge du Marketing et Commercial. Travaille à temps plein sur le projet. Ex-DSI dans différentes entreprises, entrepreneur et expert sur l’Internet des Objets (depuis 2003) et les systèmes à intelligence décisionnelle distribuée.							
      \item Lilian Prud’homme - Directeur Technique (Associé) : Travaille à temps plein sur le projet. Ex-DSI et expert en Intelligence Artificielle. Ses compétences sont, le management d’équipes, le développement logiciel et les technologies SOA, ESB TALEND, ETL, ESB, MDM, BIGDATA, JAVA et Android STUDIO.
 
      \item Delphine Pautrat - Directrice Administrative et financière (Associée) : Travaille à temps partiel sur le projet. Manager expérimentée dans la finance, la comptabilité, la conduite des opérations et la gestion du changement (notamment dans l’industrie Informatique).
      \item Sid Ali ABID , stagiaire en informatique,
      \item Moi-même, Nadia MASLOUHI, stagiaire en informatique.
    
  \end{itemize}
	
							
Mon tuteur en entreprise pendant ce stage est M.Lilian PRUD’HOMME, Toutes les  étapes de développement de l’application seront validées par lui. Mes choix technologiques quant à eux, sont soumis à l’approbation de mon tuteur de stage en entreprise.

 \section{Présentation du stage}

Il s’agit d’un stage de 4 mois (du 3 Avril au 31 juillet ) qui entre dans le cadre de la première année de Master MIAGE (Méthodes informatiques appliquées à la gestion des entreprise) à l'Université de Paris ouest Nanterre la Défense, et consiste à contribuer aux évolutions de l’application KOOPT sous Android.

\subsection{Objectif de stage}

L’objectif de stage est de participer à l’évolution corrective et fonctionnelle de la version KOOPT bêta 0.8 - 0.9 en utilisant toutes les connaissances soit en programmation soit en conception et gestion de projet d’un côté et de pouvoir les améliorer et concevoir de nouvelle technologie et de nouveaux concepts d’un autre côté.
 
les différents buts du stage : 

 \begin{itemize}
     \item La compréhension du concept B-ADSc
     \item La compréhension de l’architecture de KOOPT
     \item Le développement des sous fonctionnalités de KOOPT 
     \item Débogage de l’application KOOPT
 \end{itemize}



%%% Local Variables: 
%%% mode: latex
%%% TeX-master: "isae-report-template"
%%% End: 