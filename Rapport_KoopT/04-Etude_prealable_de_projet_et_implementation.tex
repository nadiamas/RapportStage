\chapter{Etude préalable de projet et implémentation}
\label{sec:etude préalable de projet et implémentation}

\section{Présentation de KOOPT}
Koopt est une application android qui permet de faciliter la vie quotidienne des gens ou chaque personne peut prendre un rendez-vous avec son médecin, coiffeur, etc il permet même d'organiser des sorties entre amis.
toutes personne qui possède un compte koopt est dit koopter. un koopter peut etre demandeur ou receveur de sollicitation ou les deux en même temps par exemple un médecin koopter peut recevoir des sollicitation d’autres koopters (patients) comme il peut solliciter par exemple son coiffeur pour un rendez-vous,  ses amis pour une soiree, une fete ect…
 
Koopt fonctionne sur mobile, en pair-à-pair (pas de serveurs centraux), et permet :

Des services collaboratifs et performants de conciergerie… 

• L’identification en temps réel d’un ou plusieurs acteurs / ressources de confiance pouvant répondre à un besoin, via une cooptation (bouche-à-oreilles) digitalisée ; 

• Pour tout acteur ou ressource, de proposer ses services à la communauté, partager/louer ses biens ; 

• Recommander, dans son réseau, et de proche-en-proche, un offreur de services pour la qualité de ses prestations ;  
 
Des services inédits et innovants de secrétariat…

• L’organisation automatisée et personnelle de chaque agenda des acteurs, permettant leur rencontre dans le temps et l’espace ; 
 
 L’utilisation de ces fonctionnalités dans le cadre des organisations internes des entreprises privées.


\section{L’objectif de KOOPT}

Koopt permet à tout le monde de s’organiser avec tout le monde : famille, voisins, bénévoles, amis, collègues, prestataires… C’est un outil collaboratif de conciergerie et de secrétariat, entièrement automatisé, qui généralise les échanges de ‘gré à gré’. Tout en sécurisant les données personnelles, Koopt simplifie l’organisation de chacun, en permettant :

- De rechercher des partenaires, prestataires, biens… et réserver un créneau disponible en commun; 

- De planifier un nombre important de tâches ou de RDV, de manière automatisée ; dans le respect des objectifs, priorités et contraintes de chacun;  

- De gagner en productivité : plus besoin de consulter une ressource centralisée, un tiers ou d'effectuer des envois multiples de mails pour connaître les disponibilités de chacun; 

- D’assurer ou compléter un revenu, notamment pour les auto-entrepreneurs, artisans, professions libérales, étudiants.

\section{Concurrence - Valeur Ajoutée}

 - La prise de rendez-vous : Les concurrents sont : Gong, Doodle, Calendly, Julie Desk... Toutefois Koopt est la seule application dont l’intelligence est sur le seul Smartphone, et permet de gérer simultanément plusieurs négociations de RDV tout en prenant en compte, en temps réel, les contraintes ou mises-à-jour de calendriers de chacun. C’est aussi la seule à ne partager aucune donnée (protégeant ainsi la vie privée de l’utilisateur). Enfin, Koopt n’est ni une application bureautique nécessitant l’analyse sémantique d’emails, ni un système de sondages en ligne.
 
 - La recherche de partenaires (services, prestataires, ressources...) : de nombreuses applications existent pour des offres ou échanges de services (youpijob, le bon coin, bla-bla car, Jam...), mais sont, soit très spécialisées, soit peu automatisées. Toutes proposent un système de recommandation central, qui ne prend pas en compte les critères subjectifs de chaque utilisateur. Koopt, quant à elle, permet aux utilisateurs d’accéder à différents services en même temps, sans partage de données, selon un système de recommandation de type bouche-à-oreille, et permettant des coûts d’infrastructures plus faibles (serveurs).					
 
KOOPT est la seule intelligence artificielle décisionnelle distribuée, qui fonctionne de manière proactive et intuitive. 



